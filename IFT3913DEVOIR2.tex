%%%%%%%%%%%%%%%%%%%%%%%%%%%%%%%%%%%%%%%%% 
% Cleese Assignment (For Students)
% LaTeX Template
% Version 2.0 (27/5/2018)
%
% This template originates from:
% http://www.LaTeXTemplates.com 
%
% Author:
% Vel (vel@LaTeXTemplates.com)
%
% License:
% CC BY-NC-SA 3.0 (http://creativecommons.org/licenses/by-nc-sa/3.0/)
% 
%%%%%%%%%%%%%%%%%%%%%%%%%%%%%%%%%%%%%%%%%

%----------------------------------------------------------------------------------------
%	PACKAGES AND OTHER DOCUMENT CONFIGURATIONS
%----------------------------------------------------------------------------------------

\documentclass[11pt]{article}
\usepackage{multirow}
\usepackage{siunitx} 
\input{structure.tex} % Include the file specifying the document structure and custom commands

%----------------------------------------------------------------------------------------
%	ASSIGNMENT INFORMATION
%----------------------------------------------------------------------------------------

% Required
\newcommand{\assignmentQuestionName}{Question} % The word to be used as a prefix to question numbers; example alternatives: Problem, Exercise
\newcommand{\assignmentClass}{IFT3913} % Course/class
\newcommand{\assignmentTitle}{Devoir2} % Assignment title or name
\newcommand{\assignmentAuthorName}{Xiaoqian Wang /\ Isabel Leon} % Student name
\newcommand{\Matricule}{20111352 /\ 1050029} % Intructor name/time/description
\newcommand{\Date}{18 mars \ 2022} % Due date

%----------------------------------------------------------------------------------------

\begin{document}

%----------------------------------------------------------------------------------------
%	TITLE PAGE
%----------------------------------------------------------------------------------------

\maketitle % Print the title page

\thispagestyle{empty} % Suppress headers and footers on the title page

\newpage

%----------------------------------------------------------------------------------------
%	QUESTION 1
%----------------------------------------------------------------------------------------

\begin{question}

\questiontext{T1.(15\%) Visualisez chacune des métriques de l’échantillon en créant les boites à moustaches. Calculez les
informations pertinentes et décrivez les distributions.}

\answer{
   \begin{center}
	\includegraphics[width=1\columnwidth]{T1.png}
   \end{center}

   Pour chaque métrique:\\
   NCLOC : nombre de lignes de code qui ne sont pas ni vides ni commentaires \\
   On peut voir qu'il y a beaucoup de points au-dehors de la max (Potential Outliers). 
   Donc la distributions est asymitique à gauche.\\
   
   DCP : densité de commentaires (CLOC/LOC) donnée en pourcentage \\
   DCP est un métrique en pourcentage, donc il est le plus équilibré.
   La médiane est entre 50\% à 60\%, et il y a pas de outliers.  \\

   NOCom : nombre de commits (combien de fois la classe a été changée) dans l’historique git de la classe \\
   On peut voir qu'il y a beaucoup de points au-dehors de la max (Potential Outliers). 
   Donc la distributions est asymitique à gauche.\\


   WMC : la métrique de complexité Weighed Methods per Class
   On peut voir qu'il y a beaucoup de points au-dehors de la max (Potential Outliers). 
   Donc la distributions est asymitique à gauche.\\
}

 
\end{question}

%----------------------------------------------------------------------------------------
%	QUESTION 2
%----------------------------------------------------------------------------------------

\begin{question}

   \questiontext{T2. (25\%) Évaluer l’hypothèse selon laquelle les classes qui ont été modifiées plus de 10 fois sont mieux
   commentées que celles qui ont été modifiées moins de 10 fois. Décrire d’abord la conception de l’étude et
   discuter par la suite les résultats. Suivez les étapes d’une étude empirique (choix d’étude, énoncé des
   hypothèses, définition des variables, interprétation et généralisation des résultats, discussion des menaces
   à la validité).}
   
   \answer{
      Ici on va évaluer sur l'échantillon de données pour project jfreechart,
      donc c'est une étude de cas.\\
      l’hypothèse selon laquelle les classes qui ont été modifiées plus de 10 fois sont mieux
   commentées que celles qui ont été modifiées moins de 10 fois. \\
      Des variables:
       On peut utiliser est le DCP car la dictribution de DCP est plus équilibré
     pour mesurer entre des classes.
       On peut aussi utilser le NoCom pour diviser des donnée en deux groupes.\\
      \begin{center}
         \includegraphics[width=1\columnwidth]{T2.png}
         \end{center}
      On peut voir que la moyen de DCP des classes qui ont été modifiées 
      moins de 10 fois est plus grand que celles qui ont été modifiées plus de 10 fois. \\
      Donc l’hypothèse pour ce cas n'est pas vraie.\\
      
      la nombre de classes dans groupe de moins de 10 fois est 524. \\
      la nombre de classes dans groupe de plus de 10 fois est 138. \\
      Tous les deux groupes sont suffisamment grands (>30) donc comparer la moyenne de DCP est bon. 
   
      }
   
    
   \end{question}


%----------------------------------------------------------------------------------------
%	QUESTION 3
%----------------------------------------------------------------------------------------

\begin{question}

   \questiontext{T3. (15\%) Étudier les corrélations entre NCLOC et WMC, DCP et WMC, NOCom et WMC. Visualisez les
   données, les droits de régression, etc., et expliquez pourquoi (ou pourquoi pas) ces visualisations sont
   significatives (ou pas). Dans cette étape, vous ne prenez pas de décisions: vous explorez et vous étudiez
   l'ensemble de données. }
   
   \answer{
      \begin{center}
         \includegraphics[width=1\columnwidth]{T3.png}
         \end{center}
         La corrélation entre NCLOC et WMC:\\
         On peut voir que la corrélation est forte,linéaire et positive, 
         il n’y pas de point de biais, car plus de code de logique, plus de complexité.\\
         
         La corrélation entre DCP et WMC:\\
         La corrélation est négative, mais pas si forte, car il y a beaucoup de points de biais, donc la corrélation n’est pas significativement  linéaire.
         La raison est que DCP est en pourcentage.\\

         La corrélation entre NOCom et WMC:\\
         On peut voir que la corrélation est forte,linéaire et positive, 
         il y a quelques points de biais, mais la  plupart est autour de la droite, 
         car plus de code de logique sont ajoutés, plus de complexité.\\
   }
\end{question}

%----------------------------------------------------------------------------------------
%	QUESTION 4
%----------------------------------------------------------------------------------------

\begin{question}

   \questiontext{
      T4. (30\%) Évaluer les hypothèses suivantes :
      a. WMC est une fonction linéaire du NCLOC
      b. WMC est une fonction linéaire du DCP
      c. WMC est une fonction linéaire du NOCom
      Décrire d’abord la conception de l’étude (comme en T2) et discuter les résultats par la suite. 
   }
   
   \answer{
      TODO:
   }
\end{question}

%----------------------------------------------------------------------------------------
%	QUESTION 5
%----------------------------------------------------------------------------------------

\begin{question}

   \questiontext{
      T5. (5\%) Décrivez vos conclusions dans un court paragraphe. 
   }
   
   \answer{
      TODO:
   }
\end{question}


%----------------------------------------------------------------------------------------
%	QUESTION 1
%----------------------------------------------------------------------------------------

% \begin{question}

% \questiontext{What is the airspeed velocity of an unladen swallow?}

% \begin{center}
% 	\includegraphics[width=0.5\columnwidth]{swallow.jpg} % Example image
% \end{center}

% \answer{While this question leaves out the crucial element of the geographic origin of the swallow,\
% according to Jonathan Corum, an unladen European swallow maintains a cruising airspeed velocity of 
%\textbf{11 metres per second}, or \textbf{24 miles an hour}. The velocity of the corresponding African swallows requires further research as kinematic data is severely lacking for these species.}

% \end{question}


% %----------------------------------------------------------------------------------------
% %	QUESTION 3
% %----------------------------------------------------------------------------------------

% \begin{question}

% \questiontext{Identify the author of Equation \ref{eq:bayes} below and briefly describe it in English.}

% \begin{equation}\label{eq:bayes}
% 	P(A|B) = \frac{P(B|A)P(A)}{P(B)}
% \end{equation}

% \answer{Lorem ipsum dolor sit amet, consectetur adipiscing elit. Praesent porttitor arcu luctus, imperdiet urna iaculis, mattis eros. Pellentesque iaculis odio vel nisl ullamcorper, nec faucibus ipsum molestie. Sed dictum nisl non aliquet porttitor. Etiam vulputate arcu dignissim, finibus sem et, viverra nisl. Aenean luctus congue massa, ut laoreet metus ornare in. Nunc fermentum nisi imperdiet lectus tincidunt vestibulum at ac elit. Nulla mattis nisl eu malesuada suscipit.}

% \end{question}

% %----------------------------------------------------------------------------------------

% \assignmentSection{Bonus Questions}

% %----------------------------------------------------------------------------------------
% %	QUESTION 4
% %----------------------------------------------------------------------------------------

% \begin{question}

% \questiontext{The table below shows the nutritional consistencies of two sausage types. Explain their relative differences given what you know about daily adult nutritional recommendations.}

% \begin{table}[h]
% 	\centering % Centre the table
% 	\begin{tabular}{l l l}
% 		\toprule
% 		\textit{Per 50g} & Pork & Soy \\
% 		\midrule
% 		Energy & 760kJ & 538kJ\\
% 		Protein & 7.0g & 9.3g\\
% 		Carbohydrate & 0.0g & 4.9g\\
% 		Fat & 16.8g & 9.1g\\
% 		Sodium & 0.4g & 0.4g\\
% 		Fibre & 0.0g & 1.4g\\
% 		\bottomrule
% 	\end{tabular}
% \end{table}

% \answer{Lorem ipsum dolor sit amet, consectetur adipiscing elit. Praesent porttitor arcu luctus, imperdiet urna iaculis, mattis eros. Pellentesque iaculis odio vel nisl ullamcorper, nec faucibus ipsum molestie. Sed dictum nisl non aliquet porttitor. Etiam vulputate arcu dignissim, finibus sem et, viverra nisl. Aenean luctus congue massa, ut laoreet metus ornare in. Nunc fermentum nisi imperdiet lectus tincidunt vestibulum at ac elit. Nulla mattis nisl eu malesuada suscipit.}

% \end{question}

% %----------------------------------------------------------------------------------------
% %	QUESTION 5
% %----------------------------------------------------------------------------------------

% \begin{question}

% \lstinputlisting[
% 	caption=Luftballons Perl Script, % Caption above the listing
% 	label=lst:luftballons, % Label for referencing this listing
% 	language=Perl, % Use Perl functions/syntax highlighting
% 	frame=single, % Frame around the code listing
% 	showstringspaces=false, % Don't put marks in string spaces
% 	numbers=left, % Line numbers on left
% 	numberstyle=\tiny, % Line numbers styling
% 	]{luftballons.pl}

% %--------------------------------------------

% \begin{subquestion}{How many luftballons will be output by the Listing \ref{lst:luftballons} above?} % Subquestion within question

% \answer{99 luftballons.}

% \end{subquestion}

% %--------------------------------------------

% \begin{subquestion}{Identify the regular expression in Listing \ref{lst:luftballons} and explain how it relates to the anti-war sentiments found in the rest of the script.} % Subquestion within question

% \answer{Lorem ipsum dolor sit amet, consectetur adipiscing elit. Praesent porttitor arcu luctus, imperdiet urna iaculis, mattis eros. Pellentesque iaculis odio vel nisl ullamcorper, nec faucibus ipsum molestie. Sed dictum nisl non aliquet porttitor. Etiam vulputate arcu dignissim, finibus sem et, viverra nisl. Aenean luctus congue massa, ut laoreet metus ornare in. Nunc fermentum nisi imperdiet lectus tincidunt vestibulum at ac elit. Nulla mattis nisl eu malesuada suscipit.}

% \end{subquestion}

% %--------------------------------------------

% \end{question}

%----------------------------------------------------------------------------------------

\end{document}
